%  -*- mode: LaTeX; fill-column: 78; coding: utf-8; -*-
%%%%%%%%%%%%%%%%%%%%%%%%%%%%%%%%%%%%%%%%%%%%%%%%%%%%%%%%%%%%%%%%%%%%%%%%%%%%%%
% Writing --- and Publishing -- Novels with LaTeX
% By Tammy Cravit, tammymakesthings@gmail.com
%
% Version     : 1.00
% Created at  : <Thu Sep 30 06:07:37 2021>
% Last updated: Time-stamp: <2021-10-14 09:22:04 tammy>
%%%%%%%%%%%%%%%%%%%%%%%%%%%%%%%%%%%%%%%%%%%%%%%%%%%%%%%%%%%%%%%%%%%%%%%%%%%%%%
% File Name   : chapters/02_tooling.tex
% Content     : Chapter 2: Setting Up the Tools
%%%%%%%%%%%%%%%%%%%%%%%%%%%%%%%%%%%%%%%%%%%%%%%%%%%%%%%%%%%%%%%%%%%%%%%%%%%%%%
% Copyright (c) 2021, Tammy Cravit. All rights reserved.  This work is
% licensed under a Creative Commons Attribution-NonCommercial-ShareAlike 4.0
% International License.
%
% You are free to:
%
%     Share — copy and redistribute the material in any medium or format
%
%     Adapt — remix, transform, and build upon the material
%
%     The licensor cannot revoke these freedoms as long as you follow the
%     license terms.
%
% Under the following terms:
%
%     Attribution — You must give appropriate credit, provide a link to the
%     license, and indicate if changes were made. You may do so in any
%     reasonable manner, but not in any way that suggests the licensor
%     endorses you or your use.
%
%     NonCommercial — You may not use the material for commercial purposes.
%
%     ShareAlike — If you remix, transform, or build upon the material, you
%     must distribute your contributions under the same license as the
%     original.
%
%     No additional restrictions — You may not apply legal terms or
%     technological measures that legally restrict others from doing anything
%     the license permits.
%%%%%%%%%%%%%%%%%%%%%%%%%%%%%%%%%%%%%%%%%%%%%%%%%%%%%%%%%%%%%%%%%%%%%%%%%%%%%%

\chapter{Setting Up the Tools} \label{chap:tooling}

Before we can start actually writing, we need to get some tooling set up. This
is perhaps the biggest downside of the whole process: You've got to be willing
to spend some time and energy on tooling. It's not as straightforward to get
up and running as simply downloading Word or opening a Google Doc.

For me, the investment in tooling is worth it. For you, it might not be. But
assuming you're with me so far and willing to make the investment, here's what
you'll need to get going:

\begin{description}

\item [A good text editor]. As I mentioned earlier, \LaTeX{} documents are plain
  text files. You can edit them with just about any editor you like. (Even
  Word, though this has complications I don't recomemnd. If you don't already
  have a text editor you prefer to use, here are a couple of good options:

  \begin{description}
    \item [\href{https://code.visualstudio.com}{Visual Studio Code}] --- A
      great, free, cross-platform (Windows, Mac, and Linux) editing suite with lots
      of optional extensions for all kinds of programming languages and
      editing
      tasks. \href{https://pwsmith.github.io/2020/06/05/setting-up-a-text-editor-for-latex-vscode/}{This
        article} talks about setting VSCode up for \LaTeX{} editing.
    \item [\href{https://www.sublimetext.com}{Sublime Text}] --- Another
      cross-platform editor with great add-on support for \LaTeX. \$40 and, in
      my opinion, worth every penny.
    \item [\href{https://notepad-plus-plus.org}{Notepad++}] --- An excellent
      Windows editor for text files, similar to the Notepad program that comes
      with Windows but with lots of added features.
    \item [\href{https://www.barebones.com/products/bbedit/}{BBEdit}] --- A
      storied Mac text editor (it's been around since the 1990s), with a great
      feature set and a clean, easy-to-use interface. The basic version is
      free, and you can unlock the full feature set for \$50.
    \item [\href{http://www.tug.org/texworks/}{TeXworks}] --- Included with
      the TeXLive and MacTeX distributions, and includes easy previewing of
      \LaTeX{} files. I find it to be a bit ``cluttered'' to work with, but it's
      worth giving it a try.
    \item [\href{https://www.vim.org}{Vim} and
      \href{https://www.gnu.org/software/emacs/}{Emacs}] --- These are
      designed for programmers, so they're powerful but less easy to use than
      some of the other options listed above. I personally use Emacs, because
      it's been my editor of choice since college, but it definitely has a
      steeper learning curve.
  \end{description}

  If you don't have a preference already, I recommend starting with VSCode or
  Sublime Text.

\item [The \TeX{} and \LaTeX{} Toolset]. If you're on a Linux system, this is easy
  to install. Just use your Linux distribution's package manager to install
  \href{https://tug.org/texlive/}{TeXLive}. For Windows users, follow the
  \href{https://tug.org/texlive/windows.html}{TeXLive for Windows}
  installation instructions. Mac users should follow the
  \href{https://tug.org/mactex/}{MacTeX} installation instructions.

\item [Windows Subsystem for Linux (WSL)]. The \TeX{} and \LaTeX{} tools, as well
  as Perl, were designed to be used from a command prompt. Heck, they were
  designed before today's graphical user interfaces even existed! Linux and
  Mac users have access to fully-featured command line environments with all
  the ancillary tools you'll need. Windows users should install the Microsoft
  Windows Subsystem For Linux, version 2 (WSL2), to get those tools.
  \href{https://insaid.medium.com/setting-up-wsl-2-in-windows-10-87e819d08d2e}{This
    document} explains how. You'll want to install the
  \href{https://www.microsoft.com/en-us/p/windows-terminal/9n0dx20hk701}{Windows
    Terminal} too.

\item [Perl and LaTeXML] \href{https://www.perl.org/}{Perl} is a programming
  language and tool that is available on every platform. My build environment
  uses a Perl-based tool called \href{https://dlmf.nist.gov/LaTeXML/}{LaTeXML}
  to convert the \LaTeX{} source files into a format that can be used to build
  the ebook file. 

\item [Git]. Optional, but recommended. Git is a version control system for
  programmers, but it can handle any text-based files, including your \LaTeX
  files. Think of it as a (mostly) automatic ``track changes'' for your work.

\end{description}

Once you've gotten everything installed, you're ready to start working. But
first, a quick crash course on \LaTeX{} is in order.

%%% Local Variables:
%%% fill-column: 78
%%% TeX-master: "../latex_tooling_fiction_writing"
%%% End:
