%%%%%%%%%%%%%%%%%%%%%%%%%%%%%%%%%%%%%%%%%%%%%%%%%%%%%%%%%%%%%%%%%%%%%%%%%%%%%%
% Writing --- and Publishing -- Novels with LaTeX
% By Tammy Cravit, tammymakesthings@gmail.com
%
% Version     : 1.00
% Created at  : <Thu Sep 30 06:07:37 2021>
% Last updated: Time-stamp: <2021-10-14 10:25:18 tammy>
%%%%%%%%%%%%%%%%%%%%%%%%%%%%%%%%%%%%%%%%%%%%%%%%%%%%%%%%%%%%%%%%%%%%%%%%%%%%%%
% File Name   : chapters/03_crashcourse.tex
% Content     : Chapter 3: A LaTeX Crash Course
%%%%%%%%%%%%%%%%%%%%%%%%%%%%%%%%%%%%%%%%%%%%%%%%%%%%%%%%%%%%%%%%%%%%%%%%%%%%%%
% Copyright (c) 2021, Tammy Cravit. All rights reserved.  This work is
% licensed under a Creative Commons Attribution-NonCommercial-ShareAlike 4.0
% International License.
%
% You are free to:
%
%     Share — copy and redistribute the material in any medium or format
%
%     Adapt — remix, transform, and build upon the material
%
%     The licensor cannot revoke these freedoms as long as you follow the
%     license terms.
%
% Under the following terms:
%
%     Attribution — You must give appropriate credit, provide a link to the
%     license, and indicate if changes were made. You may do so in any
%     reasonable manner, but not in any way that suggests the licensor
%     endorses you or your use.
%
%     NonCommercial — You may not use the material for commercial purposes.
%
%     ShareAlike — If you remix, transform, or build upon the material, you
%     must distribute your contributions under the same license as the
%     original.
%
%     No additional restrictions — You may not apply legal terms or
%     technological measures that legally restrict others from doing anything
%     the license permits.
%%%%%%%%%%%%%%%%%%%%%%%%%%%%%%%%%%%%%%%%%%%%%%%%%%%%%%%%%%%%%%%%%%%%%%%%%%%%%%

\chapter{A \LaTeX Crash Course} \label{chap:crashcourse}

\TeX{} (pronounced ``teKH'' with the KH being like the Scottish ``loch'') is a
document typesetting engine developed in the 1970s by Stanford University
computer science professor Donald Knuth. Knuth was looking to publish his
three-volume collection, \emph{The Art of Computer Programming}, and
determined that the typesetting tools of the time weren't up to taking
math-rich documents and making them pretty. So, as computer people are wont to
do, he decided to build his own typesetting tool.

\TeX{} is an incredibly powerful tool, with the ability for users to define
their own commands and extend the tool in nearly any way imaginable. The price
of that power is that it's also very complicated and difficult for beginners
to use. In 1984, a computer scientist named Leslie Lamport decided to build a
set of \TeX{} commands to make it easier for ``ordinary'' users to create their
own documents. This set of commands became \LaTeX{} (LAH-tekh or LAY-tekh). The
current version of \LaTeX{} is called \LaTeX2e{}.

This section of the document will provide a \emph{very} brief introduction to
some of the features of \TeX{} and \LaTeX{} as they apply to our work here. It is
emphatically \emph{not} a comprehensive guide. The canonical guide to \LaTeX
is \cite{latex2e}, and \cite{lshort} is a good (and free) beginner's guide.

\section{What Does a LaTeX File Look Like?} \label{sec:example}

Before we go much further, it's worth taking the time to show you a little bit
of what a \LaTeX{} file looks like. Here's a simple example of a book formatted
in \LaTeX:

\begin{lstlisting}
\documentclass[12pt,twoside]{memoir}
\usepackage{tikz}
\usepackage{palatino}

\begin{document}
  \title{My Awesome Book} 
  \author{Tammy Cravit}
        
  \chapter*{First Chapter} 
  \label{ch01}

  It was the \emph{best} of times, 
  it was the \textbf{worst} of times.

  These are some words:

  \begin{itemize}
    \item Grandiloquent
    \item Obsequious
    \item Idempotent.
  \end{itemize}

  \chapter*{Another Chapter} 
  \label{ch02}

  Four score and seven years ago...

  \begin{center}
    This is some centered text.
  \end{center}

  Well, I heard there was a secret chord 
  that David played and it pleased the
  Lord...
\end{document}
\end{lstlisting}

There's a lot going on here, but don't worry. The important thing to know right
here is that your written words are mixed in with \TeX{} and \LaTeX{} commands
that tell \LaTeX{} how to format the document.

If you wanted to actually format this document and turn it into a PDF file,
you could type the above commands into your text editor, save it as
\texttt{mynovel.tex}, and then typeset it with a command like this:

\begin{minipage}{\textwidth}
\texttt{pdflatex mynovel.tex}
\end{minipage}

This would produce the output file \texttt{mynovel.pdf}, and a whole bunch of
``intermediate'' files used by \LaTeX{} during the processing operation. There
are often several commands needed to fully convert a \LaTeX{} file to the
various output formats, and we'll talk about those in sections
\ref{chap:process} and \ref{sec:makefiles}.

In my example, you'll notice a number of words which begin with a
\texttt{\textbackslash} character. These are \TeX{} and \LaTeX{} commands. For
example, the command \texttt{\textbackslash{}title} sets the title of the
document, and \texttt{\textbackslash{}textbf\{\}} boldfaces the words inside
the curly braces. You'll learn some of the more essential \LaTeX{} commands in
the following sections.

\section{A Trap For Young Players} \label{sec:special_chars}

One consequence of \LaTeX{} being a markup language --- that is, one that
intersperses commands with your text --- is that there are some characters
which have to be treated with special care in your typing. These include the
ampersand, percent sign and dollar sign, as well as the curly bracket
characters and backslash. THese have special meaning for \LaTeX{}, so you have
to ``escape'' them when they appear in your text. Here's how to type them:

\begin{tabularx}{\linewidth}{|l|l|}
\caption{Escaping Special Characters} \\
\hline
\textbf{To get...}          & \textbf{You need to type...}                 \\
\hline
\endfirsthead
\caption{Escaping Special Characters (continued)} \\
\hline
\textbf{To get...}          & \textbf{You need to type...}                 \\
\hline
\endhead
Ampersand (\&)              & \texttt{\textbackslash\&}                    \\
Percent sign (\%)           & \texttt{\textbackslash\%}                    \\
Dollar sign (\$)            & \texttt{\textbackslash\$}                    \\
Curly brackets (\{...\} )   & \texttt{\textbackslash\{}...\texttt{\textbackslash\}} \\
Backslash (\textbackslash)  & \texttt{\textbackslash{}textbackslash\{\}}       \\
Underscore (\_)             & \texttt{\textbackslash{}\_}                  \\
\hline
\end{tabularx}

There are also some characters that you \emph{can} type normally, but that
will be typeset better if you type them using \LaTeX{}'s special
conventions. These include:

\begin{tabularx}{\linewidth}{|l|l|}
\caption{Typesetting Punctuation Characters} \\
\hline
\textbf{To get...}          & \textbf{You need to type...}   \\
\hline
\endfirsthead
\caption{Typesetting Punctuation Characters (continued)} \\
\hline
\textbf{To get...}          & \textbf{You need to type...}   \\
\hline
\endhead
Single quotes (`...')       & \texttt{\`{}}...\texttt{\'{}}           \\
Double quotes (``...'')     & \texttt{\`{}\`{}}...\texttt{\'{}\'{}}   \\
En dash (--)                & \texttt{-{}-{}} or \texttt{\textbackslash{}textendash\{\}}    \\
Em dash (---)               & \texttt{-{}-{}-{}} or \texttt{\textbackslash{}textemdash\{\}} \\
Ellipsis (...)              & \texttt{...}          \\
\hline
\end{tabularx}

The \href{https://latex.wikia.org/wiki/List_of_LaTeX_symbols}{\LaTeX{} Wiki}
contains a comprehensive list of special symbols and how to type them. In time
you'll find that typing these special symbols becomes second nature.xs

\section{The \LaTeX{} Document Preamble} \label{sec:preamble}

The ``content'' of your document (lines 4--19 of my example above) is
contained between the \texttt{\textbackslash{}begin\{document\}} and
\texttt{\textbackslash{}end\{document\}} commands. \TeX{} users would say the
content is contained ``inside the \texttt{document} environment'', which is
another way of expressing the same thing. In section \ref{sec:environments}
we'll look at a few other \LaTeX{} environments we commonly use. So what are the
other lines (lines 1--3 of the example)?

This section of the document is called the \emph{preamble}, and they're
responsible for setting up the document, defining other helper commands we'd
like to use, and so forth. In my current novel template, the preamble is
almost 300 lines of \TeX code before I write a word! But don't let that
intimidate you. We'll talk about some of what goes there, and I'll share my
full project template to get you started.

In the sample above, the first thing you'll see in the preamble is the
\texttt{\textbackslash{}documentclass\{\}} directive. Every \LaTeX{} file has
one, and it tells \LaTeX{} what kind of document it is: A letter, article, book,
report, etc. The elements inside the \texttt{[...]} characters are
\emph{optional arguments} to the \texttt{\textbackslash{}documentclass\{\}}
command, and in our example they tell \LaTeX{} to use a 12 point font, 8.5 x 11
inch paper, and to format for double-sided printing.

We're also telling \LaTeX{} that our document is a \texttt{memoir} document.
\href{https://www.ctan.org/tex-archive/macros/latex/contrib/memoir}{memoir} is
a class designed for typesetting poetry, fiction, non-fiction and mathematical
documents, and it provides very flexible layout and design options.

In line 2, we load the \texttt{tikz} \LaTeX{} package, a powerful add-on for
creating graphics in your \LaTeX{} document. A \emph{package} in \TeX{} is a
collection of commands that extends the capabilities of the typesetting engine
in various ways or provides new commands. There are thousands of \TeX{} and
\LaTeX{} packages to do all sorts of things; the \textbf{Comprehensive \TeX
Archive Network} (\href{https://www.ctan.org}{CTAN}) provides an index
and a network of download sites for these packages, and many of them are
already installed by TeXLive and MacTex.

Line 3 of our example loads the \texttt{palatino} package, which changes our
document font. We'll discuss font selection later. But it's important to note
that \LaTeX{} commands mostly focus on the \emph{structure} of your document,
leaving the presentation up to the \TeX{} engine. You'll see more examples of
this later in this section.

\section{Document Sections} \label{sec:sectioning}

Books are not, of course, a single monolithic block of text. Books come in
parts and chapters and sections and subsections. And LaTeX lets you designate
them, and provides a \texttt{\textbackslash{}tableofcontents\{\}} command to
automatically generate a table of contents for your document if you so
desire. The commands that are available to you, and the hierarchy of how they
fit together, are:

\begin{itemize}
\item \texttt{\textbackslash{}part\{}\emph{part title}\texttt{\}}
\item \texttt{\textbackslash{}chapter\{}\emph{chapter title}\texttt{\}}
\item \texttt{\textbackslash{}section\{}\emph{section title}\texttt{\}}
\item \texttt{\textbackslash{}subsection\{}\emph{subsection title}\texttt{\}}
\end{itemize}

The eagle-eyed might notice that in my example document, I used a slightly
different command, \texttt{\textbackslash{}chapter*\{}...\texttt{\}}. This series of
commands (you can use \texttt{\textbackslash{}part*\{}...\texttt{\}},
\texttt{\textbackslash{}section*\{}...\texttt{\}}, and
\texttt{\textbackslash{}subsection*\{}...\texttt{\}} too) behave differently in that
they don't automatically number the parts, chapters, sections and
subsections. In other words, if I write:

\Example{\texttt{\textbackslash{}chapter\{Chapter The First\}}}

\LaTeX{} will insert a chapter heading that looks something like:

\Example{\large Chapter 1\hspace{0.5in}Chapter The First \normalsize}

In contrast, if I write:

\Example{\texttt{\textbackslash{}chapter*\{Chapter The First\}}}

\LaTeX{} will insert a chapter heading that looks more like:

\Example{\large Chapter The First \normalsize}

It's a fairly common pattern in \TeX{} and \LaTeX{} for commands to have a starred
version of themselves which behaves slightly differently. Over time, you'll
learn the differences between the normal and starred versions of commands you use.

\section{Text Formatting} \label{sec:formatting}

\LaTeX{} provides a number of commands for formatting text. Here are a few of
the common ones:

\begin{tabularx}{\linewidth}{|l|l|}
  \caption{Font Formatting Commands} \\
  \hline
  Command & Function \\
  \hline
  \endfirsthead
  \caption{Font Formatting Commands (continued)} \\
  \hline
  Command & Function \\
  \hline
  \endhead
  \texttt{\textbackslash{}textbf\{}...\texttt{\}}    & Boldface text. \\
  \texttt{\textbackslash{}textit\{}...\texttt{\}}    & Italicize text. \\
  \texttt{\textbackslash{}underline\{}...\texttt{\}} & Underline text. \\
  \texttt{\textbackslash{}textsc\{}...\texttt{\}}    & Small caps text. \\
  \texttt{\textbackslash{}emph\{}...\texttt{\}}      & Emphasize text. \\
  \hline
 \end{tabularx}

You'll notice that one of these commands, \texttt{\textbackslash{}emph\{\}},
is a bit different from the others. This command \emph{emphasizes} text. You
should use the \texttt{\textbackslash{}emph\{\}} command to mark text as
emphasized, because you can more easily change the style of emphasized text
when typesetting that way. You should only use the
\texttt{\textbackslash{}textit\{\}} command if you \emph{always} want the
specified text to be italicized.

\LaTeX also provides some commands for changing the font size of blocks of
text. Again, you should be cautious about using these because they make
changing your document formatting harder. \LaTeX{} expects you to use commands
to specify the \emph{intent} behind your document, and to let it handle the
formatting. However, if you need/want to use them, here are the font size
commands:

\begin{tabularx}{\linewidth}{|l|l|}
\caption{Font Size Commands} \\
\hline
\textbf{Formatting Command} & \textbf{Resulting Text} \\
\hline
\endfirsthead
\caption{Font Size Commands (continued)} \\
\hline
\textbf{Formatting Command} & \textbf{Resulting Text} \\
\hline
\endhead
\texttt{\textbackslash{}Huge} & {\Huge Sample text} \\
\texttt{\textbackslash{}huge} & {\huge Sample text} \\
\texttt{\textbackslash{}LARGE} & {\LARGE Sample text} \\
\texttt{\textbackslash{}Large} & {\Large Sample text} \\
\texttt{\textbackslash{}large} & {\large Sample text} \\
\texttt{\textbackslash{}normalsize} & {\normalsize Sample text} \\
\texttt{\textbackslash{}small} & {\small Sample text} \\
\texttt{\textbackslash{}footnotesize} & {\footnotesize Sample text} \\
\texttt{\textbackslash{}scriptsize} & {\scriptsize Sample text} \\
\texttt{\textbackslash{}tiny} & {\tiny Sample text} \\
\hline
\end{tabularx}

You have to be careful using these font size commands for another reason: They
set the size of the font for your whole document, and that size will persist
until the next operation that changes your font size. To restrict them to just
a few words, surround \emph{the size command and the words it applies to} with
curly braces, like so:

\texttt{\{\textbackslash{}LARGE My Large Text\}}

Then the font size command will only change the specified text.

\section{Formatting Environments} \label{sec:environments}

We mentioned \emph{environments} in section \ref{sec:preamble} when we were
discussing the document preamble. An \emph{environment} is a region of text
that receives special handling when typesetting. Environments generally begin
with a \texttt{\textbackslash{}begin\{environment\}} command and end with a
corresponding \texttt{\textbackslash{}end\{environment\}} command.

Here are a couple of the most common \LaTeX{} environments you may use in your
documents.

\begin{tabularx}{\linewidth}{|l|l|}
  \caption{Common \LaTeX{} Environments}  \\
  \hline
  \textbf{Environment Name} & \textbf{Function} \\
  \hline
  \endfirsthead
  \caption{Common \LaTeX{} Environments (continued)} \\
  \hline
  \textbf{Environment Name} & \textbf{Function} \\
  \hline
  \endhead
  \texttt{flushleft}        & Left-aligns text \\
  \texttt{center}           & Centers text \\
  \texttt{flushright}       & Right-aligns text \\
  \hline
  \texttt{itemize}          & Creates a bulleted list \\
  \texttt{enumerate}        & Creates a bulleted list \\
  \texttt{description}      & Creates a description list \\
  \hline
  \texttt{math}             & Typesets math equations \\
  \texttt{quotation}        & Typesets quotations \\
  \texttt{verse}            & Typesets poetry \\
  \texttt{tabular}          & Typesets tables \\
  \texttt{picture}          & Typesets simple figures \\
  \texttt{tikz}             & Typesets diagrams \\
  \texttt{minted}           & Typesets computer code \\
  \hline
\end{tabularx}

Some of these environments take optional parameters that control the
formatting of the environment. \cite{lshort} provides a tutorial for using
these environments, so we won't discuss them in detail here.

There are also thousands of add-on \LaTeX{} packages which provide additional
capabilities.

Some \LaTeX{} packages add new capabilities, such as the \texttt{musixtex}
package for typesetting sheet music or the \texttt{chessboard} package for
printing out chess boards.

Others extend and add capabilities to existing environments, like the
\texttt{ltabularx} package, which adds extended formatting and multi-page
capabilities to the \texttt{tabular} environment).

You can find details on them with a Google search or on
\href{https://www.ctan.org}{CTAN}.

\section{Handling Errors} \label{sec:tex_errors}

If you use \LaTeX{} for long enough, you'll eventually make a mistake in your
formatting commands. It might be a missing \texttt{\}} character at the end of
a command, or something more complex. In any case, when this happens, \LaTeX{}
will produce an error and stop. The error will look something like this:

\begin{verbatim}
! LaTeX Error: \begin{center} on input line 24 
  ended by \end{document}.

See the LaTeX manual or LaTeX Companion 
 for explanation.
Type  H <return>  for immediate help.
 ...

l.31 \end{document}

?
\end{verbatim}

When you see this result, first of all, don't panic! Usually the problem is
something simple, and the error message will usually tell you what's wrong. In
my example, \LaTeX{} is telling us that we had a
\texttt{\textbackslash{}begin\{center\}} command on line 24 of the file, but
we forgot the corresponding \texttt{\textbackslash{}end\{center\}}. In this
example, \LaTeX{} got to the \texttt{\textbackslash{}end\{document\}} command
on line 31 of our file without finding the
\texttt{\textbackslash{}end\{center\}} command. If you're not sure what the
error messsage means, Google will often provide help.

When \LaTeX{} encounters an error, it will generally stop and give you a prompt
(the \texttt{?} character above). Sometimes you can just press Enter, and
\LaTeX{} will attempt to finish processing your document anyway. This will
usually produce odd results, however. A better solution is to stop \LaTeX{} by
typing an \texttt{X} and pressing the Enter key. This will return you to your
command prompt. Then you can edit the \LaTeX{} file to fix the error, and try
again.

If you're using \emph{arara} to compile your \LaTeX{} files (which we'll talk
about in section \ref{sec:arara} you won't see the \LaTeX{} command prompt,
but instead will just get a message like this:

\begin{verbatim}
  __ _ _ __ __ _ _ __ __ _
 / _` | '__/ _` | '__/ _` |
| (_| | | | (_| | | | (_| |
 \__,_|_|  \__,_|_|  \__,_|

Processing "my_novel.tex" (size: 27.5 kB, 
 last modified: 10/01/2021
07:31:08), please wait.

(PDFLaTeX) PDFLaTeX engine ..... FAILURE
Total: 1.05 seconds
\end{verbatim}

Fortunately, \LaTeX{} produces a log file when it runs; it will be named the
same as your \LaTeX{} \texttt{\.tex} file, but with a \texttt{\.log}
extension. You can edit this file to see the \LaTeX{} error message produced
by typesetting your file.

It's helpful when working on your documents to run the \LaTeX{} command often
when you're doing anything which involves adding \LaTeX{} commands to your
document --- adjusting formatting, adding an environment, defining or
customizing commands, etc. That way, when something goes wrong you'll know
which changes caused the problem.

When you're looking at the output generated by the \LaTeX{} command (in your
terminal or in the log file), two errors you'll often see call out
\texttt{underfull hbox} and \texttt{overfull hbox} conditions. A full
understanding of these errors is beyond the scope of this tutorial, but
suffice it to say that they describe situations where \LaTeX{} thinks the
typeset lines of text are either too short (undefull hboxes) or too long
(overfull hboxes). \LaTeX{} will print out the line of text that created the
condition, so you should look at your typeset document and make sure the
formatting of that line is okay. If your line is too long and has run outside
of the page's margin, you can sometimes manually fix it by breaking the
paragraph there, or by adding a manual line break with the
\texttt{\textbackslash{}newline} command.

\section{Structuring Books with \LaTeX{}} \label{sec:structuring}

As you can see from the foregoing, a simple book is just a single
\texttt{.tex} file written in your text editor, which contains all of the
commands necessary to typeset your ouvre. There's a challenge, however, with
this approach. As your work gets bigger, and your formatting demands get more
complex, this file can get very large and hard to manage.

Thankfully, \LaTeX{} provides a solution. You can split your work into multiple
files, and then combine them with the \texttt{\textbackslash{}input{}} command. The syntax
of the command is simply \texttt{\textbackslash{}input\{filename\}}. The file
name is specified relative to the current file, and you should leave off the
\texttt{\.tex} extension. So, to include the contents of the file
\texttt{ch01\.tex} from the \texttt{chapters} subfolder of your current document's
folder, you'd use the command

\Example{\texttt{\textbackslash{}input\{chapters/ch01\}}}

\section{Getting More Help} \label{sec:resources}

There are a ton of informational resources on the Internet to help you with
\LaTeX{}. Here are a few good starting places:

\begin{itemize}
\item \href{https://texfaq.org}{The \TeX{} Frequently Asked Questions List}
\item \href{https://www.ctan.org/tex-archive/info/lshort/}{A Short
    Introduction to \LaTeX2e{}}
\item \href{https://tex.stackexchange.com}{The \TeX{} Stack Exchange}
\item \cite{latex2e} and \cite{texbook}. The \TeX{}book, \cite{texbook}, will
  be helpful to you when you start trying to customize \LaTeX{} and create
  your own styles or commands, but is \emph{not} a beginner's guide.
\end{itemize}

Now that you know enough to start using \LaTeX{}, let's talk about how to set
up your writing project.

%%% Local Variables:
%%% fill-column: 78
%%% TeX-master: "../latex_tooling_fiction_writing"
%%% End:
