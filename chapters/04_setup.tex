%  -*- mode: LaTeX; fill-column: 78; coding: utf-8; -*-
%%%%%%%%%%%%%%%%%%%%%%%%%%%%%%%%%%%%%%%%%%%%%%%%%%%%%%%%%%%%%%%%%%%%%%%%%%%%%%
% Writing --- and Publishing -- Novels with LaTeX
% By Tammy Cravit, tammymakesthings@gmail.com
%
% Version     : 1.00
% Created at  : <Thu Sep 30 06:07:37 2021>
% Last updated: Time-stamp: <2021-10-14 06:31:46 tammy>
%%%%%%%%%%%%%%%%%%%%%%%%%%%%%%%%%%%%%%%%%%%%%%%%%%%%%%%%%%%%%%%%%%%%%%%%%%%%%%
% File Name   : chapters/04_setup.tex
% Content     : Chapter 4: Setting Up Your Project
%%%%%%%%%%%%%%%%%%%%%%%%%%%%%%%%%%%%%%%%%%%%%%%%%%%%%%%%%%%%%%%%%%%%%%%%%%%%%%
% Copyright (c) 2021, Tammy Cravit. All rights reserved.  This work is
% licensed under a Creative Commons Attribution-NonCommercial-ShareAlike 4.0
% International License.
%
% You are free to:
%
%     Share — copy and redistribute the material in any medium or format
%
%     Adapt — remix, transform, and build upon the material
%
%     The licensor cannot revoke these freedoms as long as you follow the
%     license terms.
%
% Under the following terms:
%
%     Attribution — You must give appropriate credit, provide a link to the
%     license, and indicate if changes were made. You may do so in any
%     reasonable manner, but not in any way that suggests the licensor
%     endorses you or your use.
%
%     NonCommercial — You may not use the material for commercial purposes.
%
%     ShareAlike — If you remix, transform, or build upon the material, you
%     must distribute your contributions under the same license as the
%     original.
%
%     No additional restrictions — You may not apply legal terms or
%     technological measures that legally restrict others from doing anything
%     the license permits.
%%%%%%%%%%%%%%%%%%%%%%%%%%%%%%%%%%%%%%%%%%%%%%%%%%%%%%%%%%%%%%%%%%%%%%%%%%%%%%

\chapter{Setting Up Your Project} \label{chap:project}

Before we begin, an important note: This is how \emph{I} set up \emph{my}
\LaTeX{} novel projects. This is the process that works for me. You might find
that a different process and structure works better for you. I encourage you
to use my guide as inspiration and a starting place. Take what works for you
and customize or change what doesn't.

\section{The Directory Structure} \label{sec:structure}

If I am writing a standalone novel, I create a folder for each book. For
series books, I create a folder for each series, and a subfolder for each
book. When I talk about the \emph{top-level folder} here, I mean the folder
for the book (standalone books) or the series (series books). If I'm referring
to the folder for a specific book in a series, I'll call it the \emph{book
folder}; for non-series books the \emph{book folder} and \emph{top-level
folder} will be synonymous.

Here's what the top-level folder for the current series book I'm writing looks
like:

Now let's look at some of these individual components and how to use
them. Although these sections will provide some guidance on how to create
these files from scratch, you can also download my book project template which
will give you a helpful starting point. See section \ref{chap:project_template}
for more details.

\section{Creating the Top-Level \LaTeX{} File} \label{sec:toplevel}

As I mentioned in section \ref{sec:structuring}, I make extensive use of
included files in my writing process. The main file for each project (which is
named \texttt{book\_\#\#\#\_book\_title.tex} for series books and
\texttt{book\_title.tex} for standalone works, contains the book's prologue
and setup commands.

\section{Chapters and Front/End Matter Files} \label{sec:files}

\section{Using a Book Journal} \label{sec:journal}

\section{Where Do I Put Other Content?} \label{sec:other}

\section{Ancillary Scripts and Tools} \label{sec:ancillary}

%%% Local Variables:
%%% fill-column: 78
%%% TeX-master: "../latex_tooling_fiction_writing"
%%% End:
